\documentclass[11pt, ]{article}

\usepackage[utf8]{inputenc}
\usepackage[a4paper, top = 2cm, left = 2cm, bottom = 2cm, right = 2cm]{geometry} 
\usepackage{hyperref}
\usepackage[anythingbreaks]{breakurl}
\usepackage{mathtools, amsthm, amssymb, amsbsy}
\usepackage[shortlabels]{enumitem}
\usepackage{xcolor}
\usepackage{fontawesome}
\usepackage{ragged2e}
\usepackage{fancyhdr}
\usepackage{lastpage}
\usepackage{enumitem}
\usepackage{changepage}

\usepackage[T1]{fontenc}
\usepackage{lmodern}

\pagestyle{fancy}
\renewcommand{\headrulewidth}{0pt}
\fancyhf{}
\rhead{\thepage\,of \pageref{LastPage}}
\rfoot{{\scriptsize Last updated on \today.}}

\linespread{1.25} 
% \pagenumbering{gobble} % Suppress page numbering

\begin{document}
	
	\begin{center}
		{\LARGE André Victor Ribeiro Amaral} \\ \vspace{6pt}
		{\small\faEnvelope~~\href{mailto:avramaral@gmail.com}{\texttt{avramaral@gmail.com~~}}} \\
		{\small\faDesktop~~\href{https://www.avramaral.com/}{\texttt{www.avramaral.com/~~~}}}	\\
		{\small\faGithubAlt~~\href{https://github.com/avramaral/}{\texttt{github.com/avramaral/}}}	
	\end{center}

{\Large \textbf{Education}}

	\begin{enumerate}
		\item \textbf{Ph.D. in Statistics}, King Abdullah University of Science and Technology. From Fall, 2020 to PRESENT. Advised by \href{https://www.paulamoraga.com/}{Dr. Paula Moraga}.
		
		\item \textbf{M.S. in Statistics}, Universidade Federal de Minas Gerais. From 02/2019 to 06/2020. Advised by \href{http://www.est.ufmg.br/~rogerwcs/}{Dr. Roger Silva}. \\
		\textit{Dissertation title}: Phase Transition Phenomenon in Percolation Models using Boolean Functions (written in Portuguese). \href{https://github.com/avramaral/MSDissertation}{\url{https://github.com/avramaral/MSDissertation}}.
		
		\item \textbf{B.S. in Statistics}, Universidade Federal de Minas Gerais. From 02/2018 to 12/2018 (Interrupted due to the admission into the Master's Program).
		
		\item \textbf{B.S. in Industrial Engineering}, Pontifícia Universidade Católica de Minas Gerais. From 02/2012 to 06/2018.
	
		%\item \textbf{Exchange Student} (by the ``Science without Borders'' Program), Curtin University. From 07/2016 to 12/2017.
	\end{enumerate}

\vspace{6pt}

{\Large \textbf{Publications}}

	\begin{enumerate}
		\item  Mahmood, M., \textbf{Amaral, A. V. R.}, Mateu, J., and Moraga, P. (2022). \textit{Modeling infectious disease dynamics: Integrating contact tracing-based stochastic compartment and spatio-temporal risk models}. Spatial Statistics. \href{https://doi.org/10.1016/j.spasta.2022.100691}{\url{https://doi.org/10.1016/j.spasta.2022.100691}}.
	\end{enumerate}

\vspace{6pt}

{\Large \textbf{Teaching}}

	\begin{enumerate}
		\item \textbf{Mini-course Instructor} in the ``Spatio-temporal Point Pattern Data Analysis with Applications in Health Surveillance and Environmental Data'' course, taught during the ``International Conference on Bioinformatics (InCoB2022).'' 11/2022. The material is available on  \href{https://avramaral.github.io/PP_tutorial/}{\url{https://avramaral.github.io/PP_tutorial/}}.
		
		\item \textbf{Graduate Teaching Assistant} in ``Applied Statistics with R'' (STAT 215), King Abdullah University of Science and Technology. Twice (Fall, 2021 and 2022). Advised by \href{https://cemse.kaust.edu.sa/people/person/joaquin-ortega-sanchez}{Dr. Joaquin Ortega}. The material is available on \href{https://avramaral.github.io/STAT215/}{\url{https://avramaral.github.io/STAT215/}}.
		
		\item \textbf{Teaching Assistant} in ``Applied Statistics and Data Analysis'' (DSA004). This was a four-day course given to ARAMCO employees in collaboration with King Abdullah University of Science and Technology. Summer, 2022. Advised by \href{https://www.paulamoraga.com/}{Dr. Paula Moraga}. The material is available on \href{https://avramaral.github.io/aramco\_course/}{\url{https://avramaral.github.io/aramco\_course/}}.
		
		\item \textbf{Graduate Teaching Assistant} in ``Contemporary Topics in Statistics'' (STAT 294), King Abdullah University of Science and Technology. Fall, 2021. Advised by \href{https://www.paulamoraga.com/}{Dr. Paula Moraga}. The material is available on \href{https://avramaral.github.io/STAT294/}{\url{https://avramaral.github.io/STAT294/}}.
		
		\item \textbf{Graduate Teaching Assistant} in ``Statistics and Probability'' (EST 031), Universidade Federal de Minas Gerais. From 02/2020 to 06/2020. Advised by \href{http://www.est.ufmg.br/~cristianocs/}{Dr. Cristiano Carvalho}. The material (written in Portuguese) is available on \href{https://avramaral.github.io/AulasEstProb/}{\url{avramaral.github.io/AulasEstProb/}}.
	\end{enumerate}

\vspace{6pt}

{\Large \textbf{Conference Presentations}}

	\begin{enumerate}
		\item \textbf{Poster presentation} at ``KAUST 2022 Workshop on Statistics.'' 11/2022. \textit{Extended Excess Hazard Model for Spatially Dependent Survival Data with Applications to Cancer Research}. The poster is available on \href{https://github.com/avramaral/AC/tree/main/KAUST_2022_STAT_WORKSHOP}{\url{https://github.com/avramaral/AC/tree/main/KAUST_2022_STAT_WORKSHOP}}.
		
		\item \textbf{Talk} and \textbf{poster presentation} at ``JSM 2022.'' 08/2022. \textit{Integrating Compartment and Point Process Models for Spatio-Temporal Modeling of Infectious Diseases}. The slides and poster are available on \href{https://github.com/avramaral/AC/tree/main/JSM\_2022}{\url{https://github.com/avramaral/AC/tree/main/JSM\_2022}}.
		
		\item \textbf{Talk} at ``GeoEnv 2022.'' 06/2022. \textit{Spatio-temporal Point Process Compartment Modeling for Infectious Diseases}. The slides are available on \href{https://github.com/avramaral/AC/tree/main/GeoEnv\_2022}{\url{https://github.com/avramaral/AC/tree/main/GeoEnv\_2022}}.
		
		\item \textbf{Poster presentation} at ``METMA X.'' 06/2022. \textit{Assessing the Effect of Model-based Geostatistics Under Preferential Sampling for Spatial Data Analysis}. The poster is available on \href{https://github.com/avramaral/AC/tree/main/METMA\_X}{\url{https://github.com/avramaral/AC/tree/main/METMA\_X}}.
		
		\item \textbf{Talk} at ``ENAR 2022.'' 03/2022. \textit{Modeling Infectious Disease Dynamics: Integrating Contact Tracing-based	Stochastic Compartment and Spatio-temporal Risk Models}. The slides are available on \href{https://github.com/avramaral/AC/tree/main/ENAR\_2022}{\url{https://github.com/avramaral/AC/tree/main/ENAR\_2022}}.
		
		\item \textbf{Poster presentation} at ``TWAS 15${}^{\text{th}}$ General Conference.'' 11/2021. \textit{Modeling Infectious Disease Dynamics: Integrating Contact Tracing-based	Stochastic Compartment and Spatio-temporal Risk Models}. The poster is available on \href{https://github.com/avramaral/AC/tree/main/TWAS\_15}{\url{https://github.com/avramaral/AC/tree/main/TWAS\_15}}.
	\end{enumerate}

\vspace{6pt}

{\Large \textbf{Honors and Awards}}

	\begin{enumerate}
		\item \textbf{CEMSE Dean's List Award}, by King Abdullah University of Science and Technology. Academic year 2021/2022.
		
		\item \textbf{Graduate Fellowship}, by King Abdullah University of Science and Technology (KAUST). From Fall, 2020 to PRESENT.
		
		The Fellowship is a competitive grant awarded to graduate students at KAUST. The grant consisted of direct research costs and living expenses. Under the Professor \href{https://www.paulamoraga.com/}{Dr. Paula Moraga}'s supervision, I have been working on the development of innovative statistical methods for geospatial data analysis with applications in health surveillance.
		
		\item \textbf{Undergraduate Scholarship}, by Brazil's ``Science without Borders'' Program. From 07/2016 to 12/2017.
		
		The Scholarship was granted to excellent students from Brazil who wanted to complete part of their undergraduate education in other Educational Institutions overseas. It covered tuition and living expenses. I completed the program as a student at Curtin University (Australia).
	
	\end{enumerate}
	
\vspace{6pt}

{\Large \textbf{Participation and Attendance}}

	\begin{enumerate}
		\item Three-week visiting period at University College London (England) under \href{https://sites.google.com/site/fjavierrubio67/}{Dr. Javier Rubio}'s supervision. 10/2022.
		
		During this time, we worked on the development of a general class of statistical models (along with inference tools) for survival data analysis. In particular, we modeled the hazard function for cancer-diagnosed patients assuming unknown causes of death (also known as ``relative survival framework'') and spatial autocorrelation.
		
		\item Gaussian Process Modeling, Design, and Optimization. Professional Development Continuing Education Course at ``JSM 2022.'' 08/2022.
		
		\item 13${}^{\text{th}}$ Summer Institute in Statistics and Modeling in Infectious Diseases (SISMID). 07/2021. I attended the following modules \vspace{-6pt}
		\begin{enumerate}[label*=\arabic*., noitemsep]
			\item Module 7: Simulation-Based Inference for Epidemiological Dynamics.
			\item Module 9: Contact Network Epidemiology.
			\item Module 12: Statistics and Modeling with Novel Data Streams.
		\end{enumerate}
		
		\item València International Bayesian Analysis Summer School, 4${}^{\text{th}}$ Edition (VIBASS4). 07/2021.
		
		\item Duke Machine Learning Virtual Summer School 2021. 06/2021.
	\end{enumerate}


\vspace{6pt}

{\Large \textbf{Miscellaneous}}

	\begin{enumerate}
		\item \textbf{Student Ambassador} in the\textit{ Computer, Electrical,  and Mathematical Science and Engineering} (CEMSE) division at King Abdullah University of Science and Technology (KAUST). Academic year 2022/2023.
		
		As a representative of the Statistics Program at KAUST, I helped in communicating the program to prospective students and answering their questions.
	\end{enumerate}




\end{document}



